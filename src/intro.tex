%!TEX ROOT=../acl2023.tex
\section{Introduction}\label{sec:intro}
\todo{}
\subsection{Main todos}
\begin{enumerate}
    \item Introduce the claim extraction problem (as a combo of check-worthiness \& hallucination-free sum)
    \item Adapt existing dataset(s) for the claim extraction task:
     \begin{itemize}
        \item CNN-Dailymail -- Each point is a pair of article and several-sentences-long \q{highlights}
        \item FEVER -- let's start from the error correction paper -- \url{https://aclanthology.org/2021.acl-long.256/}\\
        Propositional datapoints are at /mnt/data/factcheck/claim\_extraction/ csfeversum/en/0.0.1
    \end{itemize}
    \item Train a strong baseline for the problem (based on BRIO, Pegasus 2B or at least T5/BART)
    \item Maybe publish a list of novel claims extracted over some FEVER datapoints or sth? maybe blbost?
\end{enumerate}

\subsection{Todos after Marian}
\begin{enumerate}
    \item Read Šimon fully
    \item Mention interesting hallucinations (some unpunished by ROUGE):  i.e., switching the victim and perpetrator of a crime, active/passive verb, numbers are a pitfall
    \item See arrow datasets -- use dataset\_handler.py from Marian's codebase
    \item 
\end{enumerate}